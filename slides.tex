\documentclass{beamer}

\usepackage[english,spanish]{babel}
\usepackage[utf8]{inputenc}
\usepackage{graphicx}
\usepackage{wrapfig}

\usepackage{minted}
\usemintedstyle{emacs}

\usetheme{Berlin}

\setbeamertemplate{page number in head/foot}[framenumber]
\setbeamertemplate{navigation symbols}{}

% Information to be included in the title page:
\title{Elixir, La Turra II}
\author{César Guzmán Alpízar}
\institute{Vipera Ibérica}
\date{Octubre de 2020}

\begin{document}

\frame{\titlepage}

\begin{frame}
  \frametitle{Index}
  \tableofcontents
\end{frame}

\section{Actor Model and Process}
\begin{frame}
  \frametitle{Actor Model}
  \begin{itemize}
    \item Each Actor is a process
    \item Each process performs a specific task
    \item To tell a process to do something, we need to send a message
    \item Messages are pattern matched
    \item Other than that, process do not share any other information
  \end{itemize}
  \begin{center}
    \textit{From: Benjamin Tan Wei Hao. “The Little Elixir \& OTP Guidebook MEAP v10”.}
  \end{center}
\end{frame}

\begin{frame}[fragile]
  \frametitle{Spawn a process}
  \begin{minted}{elixir}
    pid = spawn(fn -> 1 + 1 end)
  \end{minted}
\end{frame}

\begin{frame}[fragile]
  \frametitle{Spawn 100 processes}
  \begin{minted}{elixir}
    Enum.each(1..100, fn _ ->
      spawn(fn -> 1 + 1 end)
    end)
  \end{minted}
\end{frame}

\begin{frame}[fragile]
  \frametitle{Receive messages}
  \begin{minted}{elixir}
    receive do
      {:pattern, n} -> IO.puts("The number is #{n}")
      _ -> IO.puts("Discarded message")
    end
  \end{minted}
\end{frame}

\begin{frame}[fragile]
  \frametitle{Receive messages}
  \begin{minted}{elixir}
    pid = spawn(fn ->
      receive do
        {:sum, a, b, pid} -> send(pid, {:result, a + b})
        _ -> IO.puts("Discarded message")
      end
    end)
  \end{minted}
\end{frame}

\begin{frame}[fragile]
  \frametitle{Send a message}
  \begin{minted}{elixir}
     send(pid, message)

     send(pid, {:sum, a, b, self()})
  \end{minted}
\end{frame}

\begin{frame}[fragile]
  \frametitle{Mantain a state with recursion}
  \begin{minted}{elixir}
  def loop(n) do
    receive do
      {:add, a} ->
        loop(n + a)

      :show_state ->
        IO.puts("The current state is #{n}")
        loop(n)

      _ ->
        IO.puts("Unrecognized error")
        loop(n)
    end
  end
  \end{minted}
\end{frame}

\section{OTP}
 \begin{frame}[fragile]
  \frametitle{GenServer}
  \begin{minted}{elixir}
defmodule Module do
  use GenServer

  # Client API
  def start_link(default) do
    GenServer.start_link(__MODULE__, default)
  end

  # Server callbacks
  def init(state) do
    {:ok, state}
  end
end
  \end{minted}
\end{frame}

\begin{frame}
  \centering
  \Large{Thanks a lot!}\\
  \hspace{0pt} \\
  \Large{Questions?}
\end{frame}

\end{document}

 %%% Local Variables:
 %%% mode: LaTeX
 %%% LaTeX-command: "latex -shell-escape"
 %%% End: